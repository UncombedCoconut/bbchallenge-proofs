\title{Correctness of the decider for cyclers\\bbchallenge}
\author{
        Tristan St\'{e}rin
}

\documentclass[a4paper,british]{article}
\usepackage{babel}
\usepackage[utf8]{inputenc}
\usepackage[margin=1in]{geometry}
%\usepackage{subfig}
\usepackage[hidelinks]{hyperref}
\usepackage{subcaption} 
\usepackage{tikz}

\usepackage{algorithm}
\usepackage[noend]{algpseudocode}

\usepackage{graphicx}

\usepackage{amsmath,amsfonts,amssymb,amsthm}


\newtheorem{theorem}{Theorem}
\newtheorem{definition}[theorem]{Definition}
\newtheorem{lemma}[theorem]{Lemma}
\newtheorem{proposition}[theorem]{Proposition}
\newtheorem{corollary}[theorem]{Corollary}

\newtheorem{observation}[theorem]{Observation}
\newtheorem{example}[theorem]{Example}
\newtheorem{remark}[theorem]{Remark}


\usepackage{microtype,xspace,wrapfig,multicol} 
\usepackage[textsize=tiny,color=lightgray]{todonotes} 
\usepackage[normalem]{ulem} % sout

\newcommand{\ts}[1]{{\color{red}#1}}
\newcommand{\tsi}[1]{\todo[inline]{TS: #1}}
\newcommand{\tsm}[1]{\todo{TS: #1}}
\newcommand{\tss}[2]{{\ts{\sout{#1}}} {\ts{#2}}}
\newcommand{\tabi}{\hspace{\algorithmicindent}}

\begin{document}
\date{}
\maketitle

\begin{abstract}
        We give the pseudocode of the decider for the ``Cyclers'' family of bbchallenge and we prove its correctness. For more context please refer to \url{https://bbchallenge.org/}. The implementation of this decider is available at: \url{https://github.com/bbchallenge/bbchallenge-deciders/}.
\end{abstract}

\section{Pseudocode}

The goal of this decider is to recognise Turing machines that cycle through the same configurations for ever where a configuration is defined by the 3-tuple: (i) state (ii) position of the head (iii) content of the memory tape. Such machines never halt. The method is simple: remember every configuration seen by a machine and return \texttt{true} if one is visited twice. A time limit (maximum number of steps) is also given for running the test in practice: the algorithm recognises any machine whose cycle fits within this limit\footnote{In practice, for machines with 5 states the decider was run with 1000 steps time limit.}.

We assume that we are given a procedure \textbf{TuringMachineStep}(machine,configuration) which computes the next configuration of a machine from the given configuration or \textbf{nil} if the machine halts at that step.


\begin{algorithm}
        \caption{{\sc decider-cylers}}

        \begin{algorithmic}[1]

                \State \textbf{struct} Configuration \{
                \State \tabi\textbf{int} state
                \State \tabi\textbf{int} headPosition
                \State \tabi\textbf{int $\boldsymbol{\to}$ int} tape
                \State \}
                \Procedure{{\sc decider-cylers}}{machine,timeLimit}
                \State \textbf{Configuration} currConfiguration = \{.state = 0,$\,$.headPosition = 0,$\,$ .tape = \{0:0\}\}
                \State \textbf{Set$\boldsymbol{<}$Configuration$\boldsymbol{>}$} configurationsSeen = \{\}
                \State \textbf{int} currTime = 0

                \While{currTime $<$ timeLimit}
                \If{currConfiguration \textbf{in} configurationsSeen}
                \State \textbf{return} true
                \EndIf
                \State configurationsSeen.\textbf{insert}(currConfiguration)

                \State currConfiguration := \textbf{TuringMachineStep}(machine,currConfiguration)
                \State currTime += 1


                \If{currConfiguration == \textbf{nil}}
                \State \textbf{return} false //machine has halted
                \EndIf
                \EndWhile

                \State \textbf{return} false
                \EndProcedure

        \end{algorithmic}
\end{algorithm}

\section{Correctness}

The set $\mathbb{N}$ denotes $\{0,1,2\dots\}$. The Turing machines that are studied in the context of bbchallenge use a binary alphabet and a single bi-infinite tape. A \textit{configuration} is defined by the 3-tuple: (i) state (ii) position of the head (iii) content of the memory tape. In the context of bbchallenge, \textit{the initial configuration} of a machine is always (i) state is 0, i.e. the first state to appear in the machine's description (ii) head's position is 0 (iii) the initial tape is all-0 -- i.e. each memory cell is containing 0. In one step, machine $\mathcal{M}$ transitions from configuration $c$ to $c'$ and we write $c \to_\mathcal{M} c'$. If the machine halts during that step we write $c \to_\mathcal{M} \bot$. By convention, $\bot \to_\mathcal{M} \bot$ is a valid transition for any machine $\mathcal{M}$.  The operator $\to_\mathcal{M}^n$ is $\to_\mathcal{M}$ applied $n\in\mathbb{N}$ times. We write $c \to c'$ when the machine is clear from context.

\begin{theorem}\label{th:main}\normalfont Let $\mathcal{M}$ be a Turing machine and $t \in \mathbb{N}$ a time limit. Let $c_0$ be the initial configuration of the machine. There exists $i\in\mathbb{N}$ and $j\in\mathbb{N}$ such that $c_0 \to^i c_i \to^j c_i$ with $i+j \leq t$ if and only if {\sc decider-cyclers}($\mathcal{M}$,$t$) returns true.
\end{theorem}
\begin{proof}
        This follows directly from the behavior of {\sc decider-cyclers}($\mathcal{M}$,$t$): all intermediate configurations below time $t$ are recorded and the algorithm returns true if and only if one is visited twice. This mathematically translates to
        there exists $i\in\mathbb{N}$ and $j\in\mathbb{N}$ such that $c_0 \to^i c_i \to^j c_i$ with $i+j \leq t$, which is what we want.
\end{proof}

\begin{corollary}\normalfont
        Let $\mathcal{M}$ be a Turing machine and $t \in \mathbb{N}$ a time limit. If {\sc decider-cyclers}($\mathcal{M}$,$t$) returns true then the behavior of $\mathcal{M}$ from all-0 tape has been decided: $\mathcal{M}$ does not halt.
\end{corollary}
\begin{proof}
        By Theorem~\ref{th:main}, there exists $i\in\mathbb{N}$ and $j\in\mathbb{N}$ such that $c_0 \to^i c_i \to^j c_i$ with $i+j \leq t$. It follows that for all $k\in\mathbb{N}$, $c_0 \to^{i+kj} c_i$. The machine never halts as it will visit $c_i$ infinitely often.
\end{proof}

\end{document}