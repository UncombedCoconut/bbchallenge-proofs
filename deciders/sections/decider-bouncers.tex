% !TeX root = ../correctness-deciders.tex

\section{Bouncers}\label{sec:bouncers}

\paragraph{Acknowledgement.} To do.


\begin{figure}[h!]
    \centering
    \includegraphics*[width=0.9\textwidth]{figures/bouncers/bouncers.pdf}
    \caption{Space-time diagrams (10,000 steps) of several \texttt{bbchallenge} \textit{bouncers}: (a) simple bouncer bouncing back and forth between expanding tape extremeties while writing 1s (b) bouncer with more complex alternating \textit{repeaters} patterns left and right of the origin (c) unilateral bouncer with a complex \textit{wall} pattern at the origin (d) unilateral bouncer, main example used throughout this section (e) bouncer entering a repetitive bouncing pattern after $\sim$6,000 steps (bottom half of the image).}\label{fig:bouncers}
\end{figure}

\subsection{Characterising bouncers}

Intuitively, a \emph{bouncer} is a Turing machine that populates a tape with
linearly-expanding patterns, called \textit{repeaters}, possibly separated or enclosed by fixed patterns called \textit{walls}. This intuitive definition corresponds to a wide range of behaviors, from simply bouncing back and forth between the tape's expanding extremities, Figure~\ref{fig:bouncers}~(a), to more complex bouncers that exhibit intricate \textit{repeaters} or \textit{wall} patterns, Figure~\ref{fig:bouncers}~(b)-(d), and that possibly start \textit{bouncing} repetitively only after a certain number of steps, Figure~\ref{fig:bouncers}~(e).

The goal of this section is to formally characterise bouncers and show that they do not halt, see Theorem~\ref{th:bouncers}.

\subsubsection{Directional Turing machines}\label{sec:bouncers:directionalTM}

We build on the concept of directional Turing machine introduced in Section~\ref{sec:conventions}. Directional Turing machines are an equivalent formulation of Turing machine where the machine head lives in between the tape cells and can point to the left or to the right. We choose here to treat $0^\infty$ as a unique symbol (instead of an infinite collection of 0s) and write $\overline{\Sigma} = \{0^\infty\}\cup\Sigma$, with $\Sigma=\{0,1\}$. For each Turing machine with set of state $S$ we introduce 2$|S|$ new configuration symbols (i.e.\ symbols used to describe machine configurations and not read/write symbols) denoting the machine head in two possible orientations, namely $\Delta = \{\lhead s | s\in S\}\cup\{\rhead s| s \in S\}$.

We define a \textit{tape}\footnote{Note that in the terminology of Section~\ref{sec:conventions}, the definition of a tape that we use here, where the head in part of the tape, corresponds to a (partial) TM configuration.} to be a finite word of the form $uhv$, where $u,v\in \overline{\Sigma}^*$ and $h\in\Delta$, moreover, $u$ and $v$ must have at most one occurrence of $0^\infty$ each, respectively as first symbol for $u$ or last symbol for $v$. We choose the initial tape to be $0^\infty \rhead{\text{A}} 0^\infty$. Now, we define tape rewrite rules which will be used to simulate a directional Turing machine which we fix from now on. Given two tapes $t$ and $t'$, a rewrite rule transforming $t$ into $t'$ will be denoted by $t\to t'$. Applying $\vdash$ successively $n\in\mathbb{N}$ times is noted $\vdash^n$. The transitive closure of $\vdash$ (i.e.\ applying $\vdash$ \textit{some} number of times) is noted $\vdash^*$.

Suppose that for $s\in S, x\in \Sigma$ we have $\delta(s,x) = (s',d,x')$ where $s'\in S, d \in \{\text{L},\text{R}\}, x' \in \Sigma$ and $\delta$ the transition function of the machine, then we define the following tape rewrite rules:


\begin{table}[h!]
    \centering
    \begin{tabular}{l|l}
        If $d = \text{L}$                               & If $d = \text{R}$ \\
        \hline
        $\begin{aligned}[t]
                 x \lhead{s}\,\,\, & \to \,\,\, \lhead{s'} x' \\
                 \rhead{s} x\,\,\, & \to \,\,\, \lhead{s'} x'
             \end{aligned}$ & $\begin{aligned}[t]
                                   x \lhead{s}\,\,\, & \to \,\,\, x' \rhead{s'} \\
                                   \rhead{s} x\,\,\, & \to \,\,\, x' \rhead{s'}
                               \end{aligned}$     \\
        \hline
    \end{tabular}\\
    \ \\ If $x=0$, also define: \\
    \begin{tabular}{l|l}
        \hline
        $\begin{aligned}[t]
                 0^\infty \lhead{s}\,\,\, & \to \,\,\, 0^\infty \lhead{s'} x'   \\
                 \rhead{s} 0^\infty\,\,\, & \to \,\,\, \lhead{s'} x'\, 0^\infty
             \end{aligned}$ & $\begin{aligned}[t]
                                   0^\infty \lhead{s}\,\,\, & \to \,\,\, 0^\infty \, x' \rhead{s'} \\
                                   \rhead{s} 0^\infty\,\,\, & \to \,\,\, x' \rhead{s'} 0^\infty
                               \end{aligned}$ \\
        \hline
    \end{tabular}\\
\end{table}

Given a tape $t=uvw$ and a word $t'=uv'w$, with $u,w\in \overline{\Sigma}^*$, $v,v' \in ( \overline \Sigma \cup \Delta)^*$, suppose that there is a rewrite rule $v \to v'$. Then $t'$ is also a tape and in this situation we write $t \vdash t'$ meaning that $t'$ is obtained from $t$ by one simulation step (i.e.\ $\vdash$ has the same meaning as for standard, non-directional, Turing machines, as introduced in Section~\ref{sec:conventions}); note that the definition is sound because, to any given tape, at most one rewrite rule applies, hence $v \to v'$ is defined uniquely and so is $t \vdash t'$. Note that the only case where $\vdash$ is not defined on a tape is if the machine halts in this configuration, i.e. an undefined transition is reached.

\begin{example}\label{ex:bouncer88427177}
    Consider \texttt{bbchallenge} machine\footnote{Accessible at \url{https://bbchallenge.org/88427177}} \#88,427,177 (Figure~\ref{fig:bouncers}~(d)), that has the following transition table:
    \[
        \begin{array}{l|ll}
              & 0          & 1          \\
            \hline
            A & 1\text{RB} & 1\text{LE} \\
            B & 1\text{LC} & 1\text{RD} \\
            C & 1\text{LB} & 1\text{RC} \\
            D & 1\text{LA} & 0\text{RD} \\
            E & \mbox{---} & 0LA
        \end{array}
    \]

    Given the above definitions, we will have the following rewrite rules with state A on the left-hand side:
    \begin{align*}
        0 \lhead{A}\,\,\,        & \to\,\,\, 1 \rhead{B}          \\
        \rhead{A} 0\,\,\,        & \to\,\,\, 1 \rhead{B}          \\
        1 \lhead{A}\,\,\,        & \to\,\,\, \lhead{E} 1          \\
        \rhead{A} 1\,\,\,        & \to\,\,\, \lhead{E} 1          \\
        0^\infty \lhead{A}\,\,\, & \to\,\,\, 0^\infty 1 \rhead{B} \\
        \rhead{A} 0^\infty\,\,\, & \to\,\,\, 1 \rhead{B} 0^\infty
    \end{align*}
    The other rewrite rules are those with left-hand side states B, C, D and E. Simulating the machine for 4 steps, starting from initial tape yields:
    $ 0^\infty \rhead{\text{A}} 0^\infty \;\vdash\; 0^\infty \, 1 \rhead{\text{B}} 0^\infty \;\vdash\; 0^\infty \, 1 \lhead{\text{C}} 1\, 0^\infty\ \;\vdash\; 0^\infty \, 1 \rhead{\text{C}} 1\, 0^\infty \;\vdash\; 0^\infty \, 1 1 \rhead{\text{C}} 0^\infty$. Hence, $0^\infty \rhead{\text{A}} 0^\infty \;\vdash^*\; 0^\infty \, 1 1 \rhead{\text{C}} 0^\infty$.
\end{example}

\subsubsection{Wall-repeater formula tapes}

We want to formalise the above-stated intuition that bouncers expand a tape by repeating \textit{repeaters} patterns between fixed \textit{walls}, and then show that bouncers do not halt, Theorem~\ref{th:bouncers}. For this, we are going to express bouncers' tapes using regular expressions over the alphabet $\Delta \cup  \overline{\Sigma}$. Given $u\in\Sigma^*$, we abbreviate the regular expression $\{u\}^*$, which represents zero or more repetitions of the word $u$, as $(u)$. Also, we write $\Sigma^+ = \Sigma^* \setminus \{ \varnothing \}$. We define a \textit{wall-repeater formula tape} to be a regular expression of the form:

\begin{align}\label{math:formulaTapes}w_1(r_1)w_2(r_2)\dots w_n(r_n) w_{n+1} h w'_1(r'_1)w'_2(r'_2)\dots w'_m(r'_m) w'_{m+1}\end{align}

if the following conditions are met:
\begin{align*}
     & n,m \geq 0,                                     \\
     & h \in \Delta,                                   \\
     & r_1,\dots,r_n,r'_1,\dots,r'_m \in \Sigma^+,     \\
     & w_2,\dots,w_{n+1},w'_1,\dots,w'_m \in \Sigma^*, \\
     & w_1, w'_{m+1} \in  \overline{\Sigma}^*
\end{align*}

and, as in the definition of a tape, $w_1$ (resp. $w'_{m+1}$) either does not contain the symbol $0^\infty$ or it starts with it (resp. ends with it). Patterns $w_i, w'_j$ are called \textit{walls} and can be empty, while the \textit{repeaters}, $r_i, r'_j$ must be nonempty. Note that we allow $n=m=0$, hence a usual tape is also a wall-repeater formula tape. From now on, for short, we say \textit{formula tapes} to mean wall-repeater formula tapes.

Given a formula tape $f$ let $\mathcal{L}(f)$ denote the language described by $f$, i.e.\ the set of tapes that match it.
% Given two formula tapes $f$ and $f'$ we will say that $f'$ is a \textit{special case} of $f$ if $\mathcal{L}(f') \subseteq \mathcal{L}(f)$. Note that it is equivalent to saying that $f'$ can be obtained from $f$ by replacing subwords of the form $(r), r\in\Sigma^*\setminus\{\varnothing\}$ by $r^n(r)r^m$ for some $n,m\geq 0$.

\begin{example}\label{ex:formulaTapes}
    Consider the formula tape $f = 0 \rhead{\text{D}} (01)$. We have $\mathcal{L}(f) = \{0\rhead{\text{D}},0\rhead{\text{D}}01,0\rhead{\text{D}}0101,\dots\}$. Consider the formula tape $f'=0^\infty(111)1110 \lhead{\text{A}} 010101(01)10^\infty$. Then: $0^\infty 1110 \lhead{A} 01010110^\infty$, $0^\infty 1111110 \lhead{A} 01010110^\infty$, and $0^\infty 1110 \lhead{A} 0101010110^\infty$ are three elements of $\mathcal{L}(f')$.
\end{example}

% \paragraph*{Aligning formula tapes.}  



% Consider the formula tape $f = 0^\infty 101(11)0 \rhead{\text{D}} 10(100)110^\infty$. The following formula tape $\tilde{f} = 0^\infty 10(11)10 \rhead{\text{D}} 101(001)10^\infty$ describes the same language as $f$, i.e.\ $\mathcal{L}(f) = \mathcal{L}(\tilde{f})$.

\paragraph*{Extending $\vdash$ to formula tapes: \textit{shift rules}.} We wish to extend the Turing machine step relation $\vdash$ (see Section~\ref{sec:bouncers:directionalTM}) to formula tapes. This is quite straightforward when the formula tape's head is pointing at a symbol of a wall (one of the $w_i, w'_j$ in \eqref{math:formulaTapes}): we simply apply a standard Turing machine step, leaving the definition of $\vdash$ unchanged.

However, we need to handle the case where the head is pointing at a repeater (one of the $r_i, r'_j$ in \eqref{math:formulaTapes}). Suppose that for some $u\in\Sigma^*$ and $r,\tilde{r}\in\Sigma^+$ and some state $s\in S$ we have $u \rhead{s} r \vdash^* \tilde{r} u \rhead{s}$. Then, for any $n\geq 0$, we have $u \rhead{s} r^n \vdash^* \tilde{r}^n u \rhead{s}$. This motivates the definition of
(right) \textit{shift rules}, rewrite rules for formula tapes, which rewrite a subword $u \rhead{s}(r)$ into $(\tilde{r})u\rhead{s}$, denoted $u \rhead{s}(r) \to (\tilde{r})u\rhead{s}$. Similarly, left shift rules are of the form $(r)\lhead{s}u \to \;\lhead{s}u(\tilde{r})$ given that $r\lhead{s}u \vdash^* \;\lhead{s}u\tilde{r}$. Note that repeaters $r$ and $\tilde{r}$ of a shift rule necessarily have the same size since $\vdash$ preserves tapes' size.

Hence, we have two cases to consider for defining $\vdash$ on the following formula tape $f$ (as defined in~\eqref{math:formulaTapes}):
$$f = w_1(r_1)w_2(r_2)\dots w_n(r_n) w_{n+1} h w'_1(r'_1)w'_2(r'_2)\dots w'_m(r'_m) w'_{m+1}$$

\begin{enumerate}
    \item (Usual step) If $h$ points to nonempty $w_{n+1}$ or $w'_1$, then $f \vdash \alpha \tilde{w}_{n+1} \tilde{h} \tilde{w}'_1\beta$ if $w_{n+1} h w'_1 \vdash \tilde{w}_{n+1} \tilde{h} w'_1$, $\tilde{w}_{n+1}, \tilde{w} \in \Sigma^*$, $\tilde{h} \in \Delta$, and $\alpha, \beta$ the above uniquely-defined beginning and ending of $f$. As for tapes, $\vdash$ is undefined if $w_{n+1} h w'_1$ corresponds to a halting configuration (i.e.\ undefined transition).
          % Using \eqref{math:formulaTapes}, write as $f=\alpha w_{n+1} h w'_1\beta$, with $\alpha$ (resp. $\beta$) the uniquely-defined beginning (resp.ending) of the formula tape $f$. Then, $f \vdash \alpha \tilde{w}_{n+1} \tilde{h} \tilde{w}'_1 \beta = f'$ if $w_{n+1} h w'_1 \vdash \tilde{w}_{n+1} \tilde{h} \tilde{w}'_1$ with $\tilde{w}_{n+1}, \tilde{w} \in \Sigma^*$ and $\tilde{h} \in \Delta$.
    \item (Shift rule) Two cases:
          \begin{enumerate}

              \item Right shift rule. If $h=\;\rhead{s\in S}$ and $w'_1$ is empty, consider the set of shift rules $\mathcal{R} = \{ u \rhead{s}(r'_1) \to (\tilde{r})u\rhead{s} \; | \; \tilde{r}\in\Sigma^+,\; u\in\Sigma^* \text{ is a suffix of } w_{n+1}\}$. If $\mathcal{R}$ is not empty then apply the right shift rule of $\mathcal{R}$ with smallest $u$ (possibly empty), call the new formula $f'$, and we have $f \vdash f'$. If $\mathcal{R}$ is empty, $\vdash$ cannot be applied on $f$.
              \item Left shift rule. If $h=\;\lhead{s\in S}$ and $w_{n+1}$ is empty, consider the set of shift rules $\mathcal{R} = \{ (r_{n})\lhead{s}u \to \;\lhead{s}u(\tilde{r}) \; | \; \tilde{r}\in\Sigma^+,\; u\in\Sigma^* \text{ is a prefix of } w'_{1}\}$. If $\mathcal{R}$ is not empty then apply the left shift rule of $\mathcal{R}$ with smallest $u$ (possibly empty), call the new formula $f'$, and we have $f \vdash f'$. If $\mathcal{R}$ is empty, $\vdash$ cannot be applied on $f$.

          \end{enumerate}

\end{enumerate}

From the above definition of $\vdash$ on a formula tape $f$, it is clear that (i) for usual tapes our new definition of $\vdash$ coincides with the old one (ii) there is at most one formula tape $f'$ such that $f \vdash f'$ and (iii) the only cases where $\vdash$ is not defined on a formula tape are either, in the usual step case, if the machine halts or, in the shift rule case, if no shift rule applies. Moreover, we get the following result:

\begin{lemma}\label{lem:vdashFormulaTapes} Let $f$ and $f'$ be formula tapes with $f \vdash f'$. Then, for all $t \in \mathcal{L}(f)$, there exists some $t' \in \mathcal{L}(f')$ such that $t \vdash^* t'$.
\end{lemma}
\begin{proof}
    If $f'$ follows from $f$ by a usual step, then applying one usual step to $t$ yields $t'\in\mathcal{L}(f')$. If $f'$ follows from $f$ by a shift rule, we apply several steps to $t$ (as many as there in the shift rule) to obtain $t' \in \mathcal{L}(f')$.
\end{proof}

% Ambiguity in shift rule application:
% If 1 s> (r) -> (1111) 1 s> and there is a 1 to the left of the initial u=1 then, we also have 11 s> (r) -> (1111) 11 s>

\begin{example}\label{ex:shiftRules}
    Taking the machine of Example~\ref{ex:bouncer88427177}, we have the right shift rule $0 \rhead{\text{D}}(01) \to (11)0\rhead{\text{D}}$. Indeed, this is because: $0 \rhead{\text{D}}01 \vdash 0\lhead{\text{A}}11 \vdash 1\rhead{\text{B}}11 \vdash 11\rhead{\text{D}}1 \vdash 110\rhead{\text{D}}$, hence $0 \rhead{\text{D}}01 \vdash^* 110\rhead{\text{D}}$, giving the shift rule. Consider the tape formula of previous Example~\ref{ex:formulaTapes}: $f' = 0^\infty(111)1110 \lhead{\text{A}} 010101(01)10^\infty$. We have $f' \vdash^{13} 0^\infty (111)1111110110 \rhead{\text{D}} (01) 10^\infty$, at this point the head points to a repeater and the set of applicable right shift rules is $\mathcal{R} = \{ 0 \rhead{\text{D}}(01) \to (11)0\rhead{\text{D}},\; 10 \rhead{\text{D}}(01) \to (11)10\rhead{\text{D}},\; 110 \rhead{\text{D}}(01) \to (11)110\rhead{\text{D}}\}$, and following the definition of $\vdash$, we apply $0 \rhead{\text{D}}(01) \to (11)0\rhead{\text{D}}$ as it has the smallest left-hand side, giving: $0^\infty (111)111111011(11)0 \rhead{\text{D}} 10^\infty$.

\end{example}

\paragraph*{Aligning formula tapes.} One last tool that we need before characterising bouncers and proving that they do not halt (Theorem~\ref{th:bouncers}) is formula tape \textit{alignment} (Definition~\ref{def:alignment}): sometimes it is necessary to rewrite a formula tape in an equivalent, \textit{aligned} form in order for being able to apply shift rules.

\begin{definition}[Alignment operator]\label{def:alignment}
    Take a formula tape, as given in~\eqref{math:formulaTapes}: $$f = w_1(r_1)w_2(r_2)\dots w_n(r_n) w_{n+1} h w'_1(r'_1)w'_2(r'_2)\dots w'_m(r'_m) w'_{m+1}$$

    The alignment operator $f \mapsto \mathcal{A}(f)$ moves repeaters away from the head $h$ by repetitively applying any of the following rules until none can longer apply:
    \begin{enumerate}
        \item Replace $(r'_{j})v$ with $v(r)$ in $f$, if $r'_j v = v r$ with $v$ a nonempty prefix of $w'_{j+1}$, $r\in\Sigma^+$, and $1 \leq j \leq m$.

        \item Replace $v(r_{i})$ with $(r)v$ in $f$, if $v r_i = r v$ with $v$ a nonempty suffix of $w_{i}$, $r\in\Sigma^+$, and $1 \leq i \leq n$.
    \end{enumerate}

    Clearly, the order application of these rules does not matter, i.e.\ $\mathcal{A}$ is well-defined and, $\mathcal{A}(\mathcal{A}(f)) = \mathcal{A}(f)$.
\end{definition}



\begin{lemma}\label{lem:sameLanguage} For any formula tape $f$, $\mathcal{L}(\mathcal{A}(f)) = \mathcal{L}(f)$, i.e.\ both $f$ and $\mathcal{A}(f)$ represent the same set of tapes.
\end{lemma}

\begin{proof}
    Consider an alignment rule as in case (1) of Definition~\ref{def:alignment}: replacing $(r'_{j})v$ with $v(r)$ if $r'_j v = v r$. Then, for all $n\in\mathbb{N},\; {r'_j}^n v = v r^n$, hence $(r'_{j})v$ and $v(r)$ describe the same language: $\mathcal{L}((r'_{j})v) = \mathcal{L}(v(r))$. Same for case (2) and for multiple applications of (1) and (2) in any order, hence we have $\mathcal{L}(\mathcal{A}(f)) = \mathcal{L}(f)$.
\end{proof}

\begin{example}\label{ex:alignment}
    Take $f=0^\infty(111)1111\rhead{\text{B}}(01)01010110^\infty$ for the machine given in Example~\ref{ex:bouncer88427177}. One can verify that no right shift rule applies to $f$ hence $\vdash$ does not apply to $f$. However, we have $ \mathcal{A}(f)=0^\infty(111)1111\rhead{B}010101(01)10^\infty$, $\mathcal{L}(\mathcal{A}(f)) = \mathcal{L}(f)$, and we can apply $\vdash$ to $ \mathcal{A}(f)$ by performing usual steps: $ \mathcal{A}(f) \vdash^{12} 0^\infty(111)1111110110 \rhead{D} (01)10^\infty$ and now the shift rule of Example~\ref{ex:shiftRules} can apply, giving $0^\infty(111)111111011(11)0 \rhead{\text{D}} 10^\infty$.
    On other alignment example is $f=0^\infty 101(11)0 \rhead{\text{D}} 10 (100) 11 0^\infty$ for which $\mathcal{A}(f) = 0^\infty 10(11)10 \rhead{\text{D}} 101(001)10^\infty$.
\end{example}





\begin{remark}[Alignment never hurts]
    Example~\ref{ex:alignment} shows that alignment (which preserves the set of recognised tapes) can allow to run a shift rule on $f'$ with $\mathcal{A}(f) \vdash^* f'$ when no shift rule was applicable directly on $f$. In fact, one can show that the alignment operator can only \textit{increase} the number of applicable shift rules: if a shift rule is applicable on $f$ then a shift rule applicable on $f'$ with $\mathcal{A}(f) \vdash^* f'$ can always be constructed.
\end{remark}

Given two formula tapes $f$ and $f'$ we will say that $f'$ is a \textit{special case} of $f$, if $f'$ can be obtained from $f$ by replacying subwords of the form $(r)$ by $r^n(r)r^m$ for some $n,m\geq 0$ and $r\in\Sigma^+$. If $f$ is a special case of $f'$ then $\mathcal{L}(f) \subseteq \mathcal{L}(f)$, the converse is open but conjectured true\footnote{See \url{https://discuss.bbchallenge.org/t/186}.} (under additional assumptions).

We finally get to the main result of this section, which characterises bouncers formally -- a bouncer is any machine on which Theorem~\ref{th:bouncers} applies:

\begin{theorem}[Bouncers]\label{th:bouncers}
    Assume we are given some Turing machine and suppose there exists a tape $t$ and some wall-repeater formula tapes $f_0,\dots,f_n, n\geq 1$, such that $0^\infty \rhead{\text{A}} 0^\infty \vdash^* t$ where $t \in \mathcal{L}(f_0)$, and $\mathcal{A}(f_i) \vdash f_{i+1}$ for $i=0,\dots,n-1$. If $f_n$ is a special case of $f_0$, then the Turing machine does not halt.
\end{theorem}

\begin{proof}
    It follows from Lemma~\ref{lem:vdashFormulaTapes} and Lemma~\ref{lem:sameLanguage} that there exists tapes $t_0, \dots, t_n$, such that $t_0 = t$ and $t_i\in \mathcal{L}(f_i)$ for $0 \leq i \leq n$, and $t_i \vdash^* t_{i+1}$ for $0 \leq i < n$, giving $t_0 \vdash^* t_n$. Since $f_n$ is a special case of $f_0$, we have $\mathcal{L}(f_n) \subseteq \mathcal{L}(f_0)$ and thus, $t_n \in \mathcal{L}(f_0)$, we can repeat this construction indefinitely and yield an infinite sequence of tapes $(t_n)_{n\in\mathbb{N}}$ such that $t \vdash^* t_i$ for all $i\in\mathbb{N}$: the machine does not halt.
\end{proof}

\begin{example}
    Theorem~\ref{th:bouncers} applies to the machine of Example~\ref{ex:bouncer88427177}, illustrated in Figure~\ref{fig:bouncers}~(d), used in our series of examples for this section. Indeed, we have $0^\infty \rhead{\text{A}} 0^\infty \vdash^{64} 0^\infty 11111101100 \rhead{\text{D}} 0^\infty$, this tape, $t=0^\infty \vdash^{64} 0^\infty 11111101100 \rhead{\text{D}} 0^\infty$ is in the language the following formula tape:
    $$f_0 = 0^\infty (111)1110(11)00\rhead{\text{D}}0^\infty$$
    At this point, and for the next 25 usual steps alignment does not affect the formulas, and we get: $$f_{25} = 0^\infty (111) 1110 (11) \lhead{\text{A}} 01010110^\infty$$
    One shift rule gives:
    $$ f_{26} = 0^\infty (111) 1110  \lhead{\text{A}} (01) 01010110^\infty$$
    After alignment:
    $$ \mathcal{A}(f_{26}) = 0^\infty (111) 1110  \lhead{\text{A}} 010101(01)10^\infty$$
    From there, $\mathcal{A}(f_{26}) \vdash f_{27}$ with:
    $$ f_{27} = 0^\infty (111) 1111  \rhead{\text{B}} 010101(01)10^\infty$$
    After 12 usual steps, not affected by alignment, we arrive at:
    $$f_{39} = 0^\infty (111)1111110110\rhead{\text{D}}(01)10^\infty$$
    One shift rule gives:
    $$f_{40} = 0^\infty (111)111111011(11)0\rhead{\text{D}}10^\infty$$
    Aligning gives:
    $$\mathcal{A}(f_{40}) = 0^\infty (111)1111110(11)110\rhead{\text{D}}10^\infty$$
    Finally, from there $\mathcal{A}(f_{40}) \vdash f_{41}$ with one usual step:
    $$f_{41} = 0^\infty (111)1111110(11)1100\rhead{\text{D}}0^\infty$$

    Now, $f_{41}$ is a special case of $f_{0}$ because $f_{41}$ differs from $f_0$ only by including one repetition of repeaters $(111)$ and $(11)$ in the walls directly to their right. The assumptions of Theorem~\ref{th:bouncers} hold and our Turing machine is a bouncer: it does not halt.
\end{example}



