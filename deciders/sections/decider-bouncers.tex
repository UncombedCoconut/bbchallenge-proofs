% !TeX root = ../correctness-deciders.tex

\section{Bouncers}\label{sec:bouncers}

Intuitively, a \emph{bouncer} is a Turing machine that builds a tape out of
ever-expanding repeating fragments. The decider analyses the machine to
identify the growing \emph{repeaters} and the \emph{walls} that separate them,
and then executes the machine symbolically to prove that this growth is endless.

\paragraph{Notation.} For a tape segment $s \in \{0, 1\}^*$, we will
write $(s)$ to mean $s$ repeated any number of times, i.e.~$s^k$ for
an unspecified $k$.

During the 

% TODO: example of a small bouncer – run a search for small numbers
% of shift rules applied

\begin{remark} The symbolic tapes we're considering can also be understood as
defining a regular language of tapes matching a particular step of the cycle
being simulated. In this interpretation, we are executing the Turing machine
on all elements of the language at the same time.
\end{remark}

Specifically, we will be considering
a tape language of the form
\begin{equation}
    \{
    0^\infty\; [0]_q\; w_0\; (r_1)^{k_1}\; w_1\; (r_2)^{k_2}\; w_2\: \cdots\: (r_n)^{k_n}\; w_n
    \mid k_1, k_2, \ldots, k_n \in \N
    \}.
\end{equation}
for a fixed list of \emph{walls} $w_0,\ \ldots,\ w_n \in \{0, 1\}^*$,
a list of \emph{repeaters} $r_1,\ \ldots,\ r_n$, and a fixed state $q$.

Fundamentally, the technique is a variant
of Closed Tape Languages, but 
To prove that a Turing machine is a bouncer,
we exhibit a set of configurations
More formally, we say that a Turing machine
is a bouncer if
We shall refine this definition later,
such that it clearly explains what machines the decider should hope to decide.


\begin{example}
    $0(110)1$ can refer to $01$, $01101$, $01101101$, and so on.
\end{example}

\begin{definition}
    A bouncer's \emph{reset configuration} is given by
    and refers to the configuration
\end{definition}

% TODO: We would probably want an illustration here. Perhaps a space-time
% diagram with the reset configurations marked?
